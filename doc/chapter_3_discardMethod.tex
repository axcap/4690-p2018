\documentclass[Report.tex]{subfiles}
\externaldocument[I-]{chapter_1_introduction}
\externaldocument[M-]{chapter_2_method}
\externaldocument[R-]{chapter_4_result}
\externaldocument[C-]{chapter_5_conclusion}
\externaldocument[RE-]{chapter_6_recognition}



\begin{document}
\chapter{Discarded Method}
\label{sec:Discarded Method}
\section{Description}
This chapter covers the methods tried but not included in the end result,
becasue they showed dissatisfactory results. This chapter is also organized
like Chapter~\ref{2-sec:Method}, but for component description please refer to
the before mentioned chapter.

\section{Text Segmentation}
\section{Preprocessing}
\subsection{Find rotation}

\begin{flushleft}
  \textbf{Approach: Hough Transform} \\
  \href{https://en.wikipedia.org/wiki/Hough_transform}{Hough transform} is a
  well known algorithm to find lines, its approach is to see if it can
  alligne some threshold of pixels on one straight line. To help with this
  process we will do a simple edge detection algorithm on the image before
  running it through the Hough Transform. \par
  Bellow, the steps needed to perform this approach are mentioned.

  \begin{enumerate}
    \item \textbf{Edge detection}
    Too make the Hough transform perform better we want to remove unnecessary
    noise. Canny edge detection algorithm is a robust and fast solution for
    this.
    \item \textbf{Line detection}
    Now perform the actuall Hough Transform.
    \item \textbf{Rotate image}
    Lastly we need to rotate the image according to the lines from the Hough
    Transform.
  \end{enumerate}
\end{flushleft}

\end{document}
