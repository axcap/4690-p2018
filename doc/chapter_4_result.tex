\documentclass[Report.tex]{subfiles}
\externaldocument[I-]{chapter_1_introduction}
\externaldocument[M-]{chapter_2_method}
\externaldocument[D-]{chapter_3_discardMethod}
\externaldocument[C-]{chapter_5_conclusion}
\externaldocument[RE-]{chapter_6_recognition}


\begin{document}
\chapter{Result - Discussion}
\label{sec:Result - Discussion}
\section{Result: Text Segmentation}
\section{Result: Preprocessing}
\subsection{Rotation}
\begin{figure}[H]
  \centering
  \includegraphics[height=4cm]{res/4angle_rot.png}
  \caption{cv.minAreaRect cannot differentiate between 0\textdegree and 180\textdegree, and 90\textdegree and 270\textdegree}
  \label{fig:4angle_rot}
\end{figure}


\subsection{Line segmentation}
\subsection{Character segmentation}
\section{Result: Classification}
\subsection{Description}
\begin{flushleft}
  Convolutional neural networks are especially good for image
  classification, because they take local spatial connections into account when
  they classify. This way it doesn't matter where in the image our
  object/character is it will be able to recognize it, same yields for rotation,
  as the CNN classifies based on local spatial connections it doesn't matter if
  the object is rotated. Hence the classification would be even more robust
  compared to the MLP.
\end{flushleft}

\section{Result: Datasets}
\subsection{Description}


\end{document}
