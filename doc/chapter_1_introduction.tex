\documentclass[Report.tex]{subfiles}
\externaldocument[M-]{chapter_2_method.tex}
\externaldocument[D-]{chapter_3_discardMethod.tex}
\externaldocument[R-]{chapter_4_result.tex}
\externaldocument[C-]{chapter_5_conclusion.tex}
\externaldocument[RE-]{chapter_6_references.tex}

\begin{document}
\chapter{Introduction}
\label{sec:Introduction}
\section{Project Description}
\textbf{The purpose of our project is to recognize text from image, and be able to do string manipulation on it. Hence this falls under the ``Optical character recognition'' (OCR) problem}

\section{Project Limitations}
As OCRs are still a challenging task even for companies like
Google, ref. reader to Googles OCR translator application on smartphones;
``Translate'', drawbacks such as; difficulty finding all the text on the photo
because of lighting, noise etc., therefore we take it for granted that we should limit our text recognition problem.

\begin{flushleft}
  \textbf{Initial limitations}
  \begin{itemize}
    \item{English alphabet [upper and lower case] + numbers [0-9] + space}
    \item{Homogeneous background, white}
    \item{Skew free text}
    \item{Computer printed text}
    \item{Even lighting}
  \end{itemize}
\end{flushleft}

\section{Project Components}
We have come to the conclusion that the OCR software has 3 main parts. Each
part is essential, for the OCR software to be able to perform its purpose.
\begin{enumerate}
 \item{\textit{Text segmentation}}
 \begin{itemize}
  \item{Finding text segments on an image and returning the text segments}
 \end{itemize}
 \item{\textit{Preprocessing}}
 \begin{itemize}
  \item{Do preprocessing on the segmented text, such as rotation and line and
  symbol segmentation. Preprocessing from definition, should be done first,
  however because of simplification we assume we manage to segment out the text
  first.}
 \end{itemize}
 \item{\textit{Classification}}
 \begin{itemize}
  \item{Classification of the symbols}
 \end{itemize}
\end{enumerate}

\begin{flushleft}
  Additionally there is one more very important component for this
  OCR software to work, \textit{\textbf{labeled data}}. Even though one might not
  need to code for this part, a good pool of labeled data is needed to be able to
  classify symbols. More on this in section \ref{Method:Datasets} \\
  4. \textit{Data classification - Gathering labeled data to train a classification algorithm}
\end{flushleft}

\section{Report Layout}
\label{subsec:Report Layout}
This report is divided into 6 sections. In
Chapter~\ref{sec:Introduction}, we introduce the project and how the report
is organized. Chapter~\ref{chap:Method} is where we present methods we have
used in our final result. Discarded methods tried will be covered in Chapter~\ref{chap:Discarded Method}.
In Chapter~\ref{chap:Result - Discussion} we review our approach and
its results, we present the confidence in our solution and what vulnerability
exist. In Chapter~\ref{chap:Conclusion} we present our thoughts on the project;
our ambition vs the actual result, and potential future improvements on the software.
Chapter~\ref{chap:Recognition} includes reference to third party
material we have used, such as; code, articles and datasets.
\end{document}
